\documentclass[10pt]{exam}
%\printanswers
\usepackage{fullpage}

\setlength{\parindent}{0pt}
\setlength{\parskip}{.25cm}

\usepackage{graphicx}

\usepackage{xcolor}

\definecolor{darkred}{rgb}{0.5,0,0}
\definecolor{darkgreen}{rgb}{0,0.5,0}
\usepackage{hyperref}
\hypersetup{
  letterpaper,
  colorlinks,
  linkcolor=red,
  citecolor=darkgreen,
  menucolor=darkred,
  urlcolor=blue,
  pdfpagemode=none,
  pdfkeywords={}
}

\definecolor{MyDarkBlue}{rgb}{0,0.08,0.45}
\definecolor{MyDarkRed}{rgb}{0.45,0.08,0}
\definecolor{MyDarkGreen}{rgb}{0.08,0.45,0.08}

\definecolor{mintedBackground}{rgb}{0.95,0.95,0.95}
\definecolor{mintedInlineBackground}{rgb}{.90,.90,1}

%\usepackage{newfloat}
\usepackage[newfloat=true]{minted}
\setminted{mathescape,
               linenos,
               autogobble,
               frame=none,
               framesep=2mm,
               framerule=0.4pt,
               %label=foo,
               xleftmargin=2em,
               xrightmargin=0em,
               startinline=true,  %PHP only, allow it to omit the PHP Tags *** with this option, variables using dollar sign in comments are treated as latex math
               numbersep=10pt, %gap between line numbers and start of line
               style=default, %syntax highlighting style, default is "default"
               			    %gallery: http://help.farbox.com/pygments.html
			    	    %list available: pygmentize -L styles
               bgcolor=mintedBackground} %prevents breaking across pages
               
\setmintedinline{bgcolor={mintedBackground}}
\setminted[text]{bgcolor={mintedBackground},linenos=false,autogobble,xleftmargin=1em}
%\setminted[php]{bgcolor=mintedBackgroundPHP} %startinline=True}
\SetupFloatingEnvironment{listing}{name=Code Sample}
\SetupFloatingEnvironment{listing}{listname=List of Code Samples}

\begin{document}

\section*{CSCE 155 - Lab 5.0 - Functions - Worksheet}

Names: \underline{\hspace{10cm}}

\begin{questions}

\question Complete the table below using 3 functions you've used prior to this lab.
\begin{center}
\begin{tabular}{|l|l|l|}
\hline
Return Type & Function Name & Parameter Type(s) \\
\hline
\hline
\mintinline{c}{double} & \mintinline{c}{pow} & \mintinline{c}{(double, double)} \\
\hline
~ & ~ & ~ \\
\hline
~ & ~ & ~ \\
\hline
~ & ~ & ~ \\
\hline
\end{tabular}
\end{center}

\question Open all the files and review the code to answer the following 
questions.
\begin{parts}
  \part Identify all of the user-defined functions in the program 
  along with the functions' return types and the number of parameters 
  they accept.
  \begin{solution}[2cm]
  ~
  \end{solution}
  \part Among the functions \mintinline{c}{main}, \mintinline{c}{strToIntArray}, \mintinline{c}{insertionSort}, and \mintinline{c}{getOrderStatistic}: identify which function calls which other function(s).
  \begin{solution}[2cm]
  ~
  \end{solution}
  
  \part How might you implement the following function?  Hint: it should only be a one-liner.
\begin{minted}{c}
int getMin(int *arr, n) {

}
\end{minted}

  \part Prototypes and documentation were placed in the header file 
  and function definitions were placed into a separate source file.
  Identify at least one advantage of this approach.
  \begin{solution}[2cm]
  ~
  \end{solution}
    
\end{parts}

\question Demonstrate your final color utility program and testing file to a lab instructor, have them sign this worksheet and turn it in.

\end{questions}
  
Lab Instructor Signature\underline{\hspace{7.5cm}}

\end{document}
